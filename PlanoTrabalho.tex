\pagestyle{empty}
\cleardoublepage
\pagestyle{fancy}

%espaco entre linhas
\onehalfspacing

\chapter{Plano de Trabalho}\label{cap7}


Os detalhes do plano de tabalho são apresentados na tabela \ref{t_cronograma}. O aluno
ingressou no programa em fevereiro de 2013 e tem previsão de término em dezembro
de 2014.

\begin{table}[!htpb]
\centering

\begin{small}

\setlength{\tabcolsep}{2pt}

\begin{tabular}{|c|c|c|c|c|c|c|c|c|c|c|c|c|c|c|c|c|c|c|c|c|c|c|c|} \hline

 & \multicolumn{11}{c|}{2013}  & \multicolumn{12}{c|}{2014}   \\ \cline{2-22}
\raisebox{1.5ex}{Etapa} & 2 & 3 & 4  & 5 & 6 & 7 & 8 & 9 & 10 & 11 & 12 & 1 & 2 & 3 & 4 & 5 & 6 & 7 & 8 & 9 & 10 & 11 & 12 \\ \hline

Disciplinas do Mestrado & X & X & X & X & X & X & X & X & X & X & X & & & & & & & & & & & & \\ \hline
Levantamento Bibliográfico &  &  &  & X & X & X & X & X & X & X & & & & & X & X & X & X & X & X & & & \\ \hline
Desenvolvimento do Algoritmo & & & & & X & X & X & X & X & X & X & X & X & X & & & & & & & & & \\ \hline
Implementação do Algoritmo & & & & & & & X & X & X & X & X & X & X & X & X & X & X & X & & & & & \\ \hline
Implementação da Comunicação & & & & & & &  &  & &  &  &  &  &  & &  &  & X & X & X & X & & \\ \hline
Experimentos & & & & & & & & & & & & & X & X & X & X & X & X & X & X & X & X & \\ \hline
Escrita da Qualificação & & &  & & & & & & & & & &  &  & & & & X & X & & & & \\ \hline
Escrita de Artigo & & & & & & & & & & & & & & & & & & X & X & X & X & & \\ \hline
Escrita da Dissertação & & & & & &  & & & & & & & & & & & X & X & X & X & X &  &  \\ \hline

\end{tabular}
\end{small}
\caption{Cronograma das atividades realizadas e previstas}
\label{t_cronograma}
\end{table}

O plano de trabalho foi dividido em várias etapas. Nas quais foram realizadas todas as disciplinas necessárias, feito um levantamento bibliográfico inicial, que direcionou a ideia inicial do algoritmo. Foi feito o desenvolvimento do algoritmo, que vem sendo realizado desde agosto de 2013, com vários estudos e tutoriais principalmente relacionados a StarPU. Foi implementado os algoritmos de comparação, que apresentam certa complexidade de implementação. E inicialmente foi gerado um algoritmo baseado no HDSS, que era uma modificação do algoritmo original. Após a familizarização com o StarPU foi possível o total desenvolvimento do algoritmo e implementação do algoritmo de escalonamento. 

Os próximos passos são a implementação do modelo de comunicação, que está sendo feito. Que vai levar em consideração a quantidade de dados transmitida entre os processadores. Também será adicionada como aplicação teste uma rede de inferencia regulatória gênica.



