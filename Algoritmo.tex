\pagestyle{empty}
\cleardoublepage
\pagestyle{fancy}

\chapter{Algoritmo Proposto}\label{cap4}

\section{Introdução}\label{cap4:intro}

Esta seção demonstra detalhes de implementação do algoritmo proposto para o balanceamento de carga dinâmico em aglomerados de GPUs.


\section{Algoritmo}\label{cap4:Algoritmo}

\section{Implementação}\label{cap4:Implementacao}


\section{IPOPT}\label{cap4:ipopt}

IPOPT (Interior Point Optimizer) é um pacote de software aberto para otimização não linear. IPOPT pode resolver problemas de programação não-linear da seguinte forma:
	


onde x $\in$ a $\Re ^ n$ são as variaveis de otimização, $f : R^n em R$ é a função objetivo, e $g: R^n$ em $R^M$ são as restrições não lineares. A função $f(x)$ e $g(x)$ podem ser linear ou não linear.  

Para resolver um problema de otimização precisa-se criar um IpoptProblem com a função CreateIpoptProblem, que posteriormente precisa ser passado para a função IpoptSolve. O IpoptProblem criado por CreateIpoptProblem contem as dimensões do problema, as variaveis e os limites das restrições.








