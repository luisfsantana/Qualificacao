\pagestyle{empty}
\cleardoublepage
\pagestyle{fancy}

\chapter{Implementação}\label{cap5}

\section{Introdução}\label{cap5:intro}

A aplicação foi feita na linguagem C, com o \emph{framework} StarPU. ~\cite{starpu}. StarPU  é uma ferramenta para programação paralela que suporta 
arquiteturas híbridas como CPUs multicore e aceleradores.  StarPU propõe 
uma abordagem de tarefas independentes baseada na arquitetura. Codelets são definidos como uma 
abstração de uma tarefa que pode ser realizada em um núcleo de uma CPU multicore ou 
submetido a um acelerador. Cada codelet pode ter várias implementações, um 
para cada arquitetura em que codelet pode ser realizada utilizando linguagens específicas 
e bibliotecas para a arquitetura alvo. Uma aplicação StarPU é descrita como  
um conjunto de Codelets com dependências de dados. 

A ferramenta tem um conjunto de políticas de escalonamento implementadas que o programador pode 
escolher de acordo com as características da aplicação. A principal delas é a 
uso do algoritmo de escalonamento estático HEFT (Heteregeneous Earliest Finish Time) para 
agendar tarefas com base em modelos de custo de execução da tarefa. 

Para cada dispositivo, como GPU ou CPU, foi programado um codelet. Um codelet é uma estrutura que representa um núcleo computacional. Tal codelet pode conter uma implementação do mesmo kernel em diferentes 
arquiteturas (por exemplo, CUDA e x86). As aplicações foram implementadas  
dividindo o conjunto de dados em tarefas. As tarefas são 
independentes, com cada tarefa a receber uma parte do conjunto de entrada. 

Para avaliar o nosso algoritmo de balanceamento de carga, nós modificamos o algoritmo de escalonamento padrão do StarPU. A modificação do algoritmo de balanceamento de carga é feita alterando a variável STARPU\_SCHED. O \emph{framework} STARPU tem uma API que permite modificar as políticas de escalonamento. Existem estruturas de dados e funções que aceleram o processo de desenvolvimento. Por exemplo, a função "double starpu\_timing\_now (void)" que retorna a data atual em micro segundos, o que torna mais fácil para a determinação de medidas de tempo de execução. 

Três outros algoritmos foram implementados para comparação: o guloso, o estático e 
HDSS. O guloso consistiu em dividir o conjunto de entrada em pedaços e atribuir 
cada pedaço da entrada a qualquer processador ocioso, sem qualquer atribuição de prioridade. O estático~\cite{raphael}, mede as velocidades de processamento antes da execução e  atribui um conjunto 
de blocos estático para cada processador, no início da execução, com os
tamanhos do bloco proporcional à velocidade do processador. Por fim, o HDSS ~\cite{HDSS}  
foi implementado utilizando a estimativa dos mínimos quadrados para estimar os pesos e 
dividido em duas fases: fase de adaptação e fase de conclusão. 

A biblioteca utilizada para resolver o sistema de equações é a IPOPT \cite{point}. IPOPT (Interior Ponto Otimizar) é um pacote de software de código aberto para otimização não-linear em grande escala. Ele pode ser utilizado para resolver problemas de programação lineares gerais.

